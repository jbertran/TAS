\documentclass[aspectratio=169]{beamer}

\usepackage{graphicx}
\usepackage{multicol}
\usepackage[utf8]{inputenc}
\usepackage{amsmath}
\usepackage{amsfonts}
\usepackage{listings}
\usetheme{Rochester}

\title{TAS -- Article Presentation \\ Verification of Device Drivers and Intelligent Controlles: a Case Study \\ \textit{David Monniaux}}
\author{Thibault Krebs \and Jordi Bertran de Balanda}
\date{Février 2017}

\begin{document}

\begin{frame}
  \titlepage
\end{frame}

\begin{frame}{Overview}
  \begin{multicols}{2}
    \tableofcontents
  \end{multicols}
\end{frame}

\section{Article background}

\subsection{Device drivers and critical systems}
\begin{frame}{Device drivers and critical systems}
    \begin{block}{Device driver}
    \begin{itemize}
        \item Extremely hard to verify in isolation - take the \textbf{hardware} reaction into account.
        \item Hardware and software interface with memory mapped registers, within a simple specification.
    \end{itemize}    
    \end{block}
    \begin{block}{Industry shift}
    \begin{itemize}
        \item Critical system devices were \textit{extremely} simple
        \item Highest security rating devices ship with limited I/O subsystem in place of an OS
        \item Progressive shift towards off-the-shelf interfaces like USB or Ethernet 
    \end{itemize}
    \end{block}
    Device driver certification is \textbf{very} desirable. To achieve it, we need to be able to prove a device driver's code is \textit{sound}.
\end{frame}

\subsection{Challenges involved}
\begin{frame}{Challenges}
Two main challenges to using consumer interfaces in critical systems:
\begin{itemize}
    \item Many unnecessary features: hotswapping, plug-and-play, dynamic configuration, etc... Unneeded in most \textbf{static} critical applications
    \item Higher bandwidth, which means the use of DMA (Direct Memory Access), making the device driver behaviour essentially \textbf{asynchronous}, therefore harder to analyze
\end{itemize}
\begin{alertblock}{Asynchronous behaviour is BAD}
Asynchronous behaviour is \textbf{especially} important in device drivers, where software/hardware interfaces \textit{and} driver state are all, in most drivers, in \textit{shared} memory space.
\end{alertblock}
\end{frame}

\section{Implementation}

\subsection{Objectives}
\begin{frame}{Drivers chosen}
    The author chose the OHCI specification for USB 1.1 devices as his testbed for device driver analysis.
    \begin{block}{Goals}
    \begin{enumerate}
        \item Develop a formal definition of the informal English specification of OHCI
        \item Extend the Astrée abstract interpreter for use with the specifics of 
        \item Model the device driver's asynchronous behaviour into something that can be analyzed by Astrée
    \end{enumerate}
    \end{block}
    We will see later that some concessions had to be made in order to have a working end result.
\end{frame}

\subsection{Abstract Interpretation}
\begin{frame}{Extensions made to Astrée}
    Main potentially problematic issues with device driver code:
    \begin{itemize}
        \item Pointer arithmetic of variable step: modelized with $ptr = base + offset$ using non-relational abstract domains.
        Since most of the structures used will be static within the ``critical system'' application, this is especially important.
        \begin{itemize}
            \item interval domain $[m, M]$ for the bounds of the pointer
            \item congruential domain $a + b.\mathbb{Z}$ for pointer stride
        \end{itemize}
        \item Extensive use of bitmaps coming from the concern of optimizing memory usage - typically in option vectors or 
        state variables. This is done with a specific domain, with no further description from the author.
    \end{itemize}
\end{frame}

\subsection{Dynamic State Partitioning}
\begin{frame}[fragile]{A perk of using Astrée: Dynamic State Partitioning}
    Instead of analyzing a code fragment for all possible input states, split the input states according to a predicate.

    For example:
    \begin{minipage}{0.5\textwidth}
    \begin{verbatim}
        if (condition) {
          a;
        }
        else {
          b;
        }
        c;
    \end{verbatim}
    \end{minipage}%
    \begin{minipage}{0.5\textwidth}
    \begin{verbatim}
        if (condition) {
          a;
          c;
        }
        else {
          b;
          c;
        }
    \end{verbatim}
    \end{minipage}
    This is especially useful when an array index variable potentially points to many different locations verifying local invariants (requiring usually complex and costly custom relational domains).
\end{frame}

\section{Driver Modeling}

\subsection{Asynchronous Behaviour Modeling}
\begin{frame}{Driver/Program Asynchronous Modeling}
  We consider the interactions of the controller and driver program a sequence $p_1a^*p_2a^*p_3a^*...$, where $p_i$ are the atomic steps of execution of the main program,
  and $a$ steps are the non-deterministic actions of the controller.

  We \textbf{ignore} the sequences that do not interfere: all we are interested are the $p_1a^*p_2a^*p_3a^*$ where $a^*$ precedes a $p_i$ that \textbf{reads or writes} into
  shared memory.
\end{frame}

\section{Results}

\subsection{Final Abstract Interpreter}
\begin{frame}{Abstract Interpretation Results}
  With the extensions, the analyzer is able to detect:
  \begin{itemize}
  \item Out of bounds memory accesses
  \item Null pointer issues and invalid pointer dereferencing
  \item Invalid pointer arithmetic
  \item Arithmetic issues - overflow, 0-division, etc..
  \end{itemize}
  \begin{exampleblock}{End result}
    The extended analyzer is able to prove correct (free of runtime errors) the USB 1.1 device driver used by the author, even in the presence of the ``done queue''.
  \end{exampleblock}
\end{frame}

\subsection{Limitations}
\begin{frame}{Limitations and Caveats}
  \begin{itemize}
  \item Code alterations:
    \begin{itemize}
    \item \texttt{goto} instructions were removed to achieve better precision.
    \item Padding in some data structures caused Astrée to allocate sufficient unneeded variables as to impact performance.
    \end{itemize}
  \item Loss of precision in data bit length because of the ``done queue'' confusing 16 bit FD for 32 bit FD
  \item Bitmap troubles: OHCI specification sometimes orders information storage in the 2 lowest-order bits of pointers that are supposed to be aligned in 4-byte stripes.
  \item The deeply nested \textit{Endpoint Descriptor} data model, when active in the driver, caused the analyzer to run over 100 times slower (6 minutes with the ``done queue'' disabled against 10 hours with it enabled)
  \end{itemize}
\end{frame}

\section{Conclusion}

\subsection{Related Work}
\begin{frame}{Related Work}
  \begin{itemize}
  \item ``Bug-finding'' techniques (, ad-hoc techniques for Linux/BSD kernels)
    \begin{itemize}
    \item Microsoft SLAM group - model checking of boolean abstractions: begin with a boolean control-flow abstraction of the program, then refine it.
    \item Ad-hoc techniques with Linux and OpenBSD kernels - statistical analysis, attempting detection of incorrect or inconsistent usage, etc..
    \end{itemize}
  \item ``Doomed'' approaches for device drivers advocating a language change:
    \begin{itemize}
    \item ``Safe'' subset of C (MISRA-C)
    \item Type-safe languages (CCured)
    \end{itemize}
  \end{itemize}
\end{frame}

\subsection{Objectives achieved}
\begin{frame}{Conclusion}
  \begin{exampleblock}{End result}
    A verification tool for proving the absence of runtime errors in a \textbf{critical} device driver.
    \begin{itemize}
    \item Some compromises are acceptable, like ignoring unneeded features (hotswapping, plug-and-play, etc..)
    \item Some compromises are natural, like using mostly static allocation in a system where the components (hardware and software) are known.
    \item Some future work on the efficiency of this approach is desirable
      \begin{itemize}
      \item through limiting the use of Dynamic State Partitioning
      \item through a better pointer abstract domain, enabling better handling of complex dynamic structures
      \end{itemize}
    \end{itemize}
  \end{exampleblock}
  \begin{alertblock}{No generic solution}
    No one-size-fits-all solution! This specific driver's proof required iterative changes to existing abstract interpreters before being possible in a single pass.
  \end{alertblock}
\end{frame}

\end{document}
