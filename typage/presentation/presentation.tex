\documentclass{beamer}

\usepackage{graphicx}
\usepackage[utf8]{inputenc}
\usepackage{amsmath}
\usepackage{listings}

\lstset{
  basicstyle=\ttfamily,
  breaklines=true,
  postbreak=\raisebox{0ex}[0ex][0ex]{\ensuremath{\hookrightarrow\space}},
  mathescape
}

\usetheme{Rochester}

\title{TAS \\ Compact Bit Encoding Schemes for Simply-Typed Lambda-Terms}
\author{Hugo Nakagawa \and Jordi Bertran de Balanda}
\date{Novembre 2016}

\begin{document}

\begin{frame}
  \titlepage
\end{frame}

\begin{frame}
  \tableofcontents
\end{frame}

\section{Compression de $\lambda$-termes}

\subsection{Compression d'ordre supérieur}
\begin{frame}{Compression d'ordre supérieur}
  \begin{itemize}
  \item
    En théorie, offre un ratio de compression \emph{très élevé} \\
    Par exemple: \\
    $a^{2n}b \equiv$ let \emph{twice} = $\lambda f.\lambda x.f(fx)$ in $twice^n a b$
  \item Produire des données compressées qui peuvent être manipulées \emph{sans décompression}
  \item Émuler des schémas de compression 'classiques'
  \end{itemize}
\end{frame}

\section{Codage typé sans étiquettes}

\subsection{Principe}
\begin{frame}{Codage typé sans étiquettes - principe}
  \begin{block}{But}
    \begin{enumerate}
      \item Retirer les étiquettes qui déterminent visuellement l'associativité des $\lambda$-termes
      \item Générer une séquence binaire contenant informations de types et indices de De Bruijn, correspondant au programme encodé
    \end{enumerate}
  \end{block}
  \pause
  \begin{block}{Outils}
    \begin{enumerate}
      \item Environnement de typage $\Gamma$
      \item Type du programme $\tau$
      \item Séquence sans étiquettes d'indices de De Bruijn
    \end{enumerate}
  \end{block}

\end{frame}

\subsection{Transformation}
\begin{frame}{Codage typé sans étiquettes - transformation}
  \begin{block}{Application}
    $$0(011)(011) \Rightarrow 0011011$$
  \end{block}
  \begin{block}{Abstraction}
    $$(\lambda_{o \rightarrow o \rightarrow o}.\lambda:o.(10)0)0 \Rightarrow \lambda_{(o \rightarrow o \rightarrow o) \rightarrow o \rightarrow o}.\lambda_{o \rightarrow o} 1 0 0 0$$
  \end{block}
  \pause
  Avec $\Gamma$ l'environnement de types associé, on \emph{garantit} qu'il est possible de retrouver
  la forme originale étiquetée. On définit \emph{formellement} et on prouve les fonctions
  d'encodage et de décodage qui permettent de passer d'un état à l'autre.
\end{frame}

\subsection{Encodage binaire}
\begin{frame}{Codage typé sans étiquettes - encodage binaire}
\begin{block}{Table de types}
\begin{itemize}
\item Types codés en notation préfixe: $[\![o]\!] = 0; [\![\tau_1 \rightarrow \tau_2]\!] = 1[\![\tau_1]\!][\![\tau_2]\!]$

    \item Séquence de types sans séparateurs, indexés dans leur ordre de rencontre
\end{itemize}
\end{block}
\pause
\begin{block}{Environnement, type du programme}
\begin{itemize}
    \item Types codés avec la même  notation
    \item Séquence de types sans séparateurs
    \item Éventuellement à recompresser 'classiquement'
\end{itemize}
\end{block}
\end{frame}

\begin{frame}{Codage typé sans étiquettes - encodage binaire}
\begin{block}{Séquence de symboles}
\begin{itemize}
    \item Chaque $\lambda_\tau$ est remplacé par l'indice de $\tau$, qu'on a préparé dans la table de types \item Chaque indice de variable est incrémenté de $k$, avec $k-1$ l'indice le plus élevé de l'environnement
    \item On peut optimiser la quantité de bits utilisés en raisonnant sur les types disponibles 
\end{itemize}
\end{block}
On encode ce résultat en binaire, donnant le résultat final.
\end{frame}

\section{Expérimentations}

\begin{frame}{Expérimentations}
\begin{block}{Méthodologie}
\begin{itemize}
\item Données brutes tirées de séquençage ADN, dictionnaires, articles et jeux de test XML
\item Génération de $\lambda$-termes avec un compresseur d'ordre supérieur
\end{itemize}
\end{block}
\pause
\begin{itemize}
\item Performances décevantes sur des petits jeux de tests dues à l'overhead d'informations de types
\item Performances globalement supérieures aux algorithmes précédemment disponibles
\end{itemize}
\end{frame}

\section{Conclusion}

\begin{frame}{Conclusion}
\begin{itemize}
\item Deux techniques de compression de $\lambda$-expressions
    \begin{itemize}
      \item Sans étiquette : mieux adaptée aux petits programmes
      \item Avec CFG : optimisée pour les expressions d'ordre supérieur
    \end{itemize}
\item Excellents résultats sur des programmes longs
\item Meilleurs résultats en compression de $\lambda$-expression d'ordre supérieur
\item Perspectives : adaptation à d'autres types de structures
\end{itemize}
\end{frame}

\end{document}
